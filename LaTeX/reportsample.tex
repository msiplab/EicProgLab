% reportsample.tex
%
% バージョン: 2017.4.7
%
% 説明:
% プログラミング演習課題提出用PDFを生成する
% サンプル TeX ファイル
%
% 作成者:村松正吾
%
% PDF作成手順:
%
% $ platex reportsample
% $ dvipdfmx reportsample
%
% 更新履歴:
% 2017.4.7 新規作成
%
\documentclass[a4paper,12pt]{jsarticle}

\usepackage{listings}

\newcommand{\mynumber}{学籍番号} % 学籍番号を記載
\newcommand{\myname}{新潟 太郎} % 氏名を記載

\newcommand{\myheader}{ % ここはそのままでよい。
\begin{flushright}
\mynumber\hspace{1zw} \myname\hspace{1zw} \today\end{flushright}}

\begin{document}

%%%%%%%%%%%%%%%%%%%%%%%%%%%%%%%%%%%%%%%%%%%%%%%%%%%%%%%%%
\section*{課題1-1(基礎)}
\myheader

\subsection*{説明}

ここに、課題(基礎)のソースコード、実行の様子の説明を記載。

\subsection*{考察}

ここに、課題(基礎)のソースコードのプログラミング手法や
コンピュータの動作などについて理解したこと、疑問に思ったこと、
残された課題と改善策などを記載。

\subsection*{ソースコード}

\subsubsection*{helloworld.c}
\begin{lstlisting}[basicstyle=\ttfamily\footnotesize, frame=single]
#include <stdio.h>

main()
{
  printf("hello, world!\n");
}
\end{lstlisting}

\subsection*{実行結果}

\begin{quote}
\begin{verbatim}
  >> hello, world!

\end{verbatim}
\end{quote}

%%%%%%%%%%%%%%%%%%%%%%%%%%%%%%%%%%%%%%%%%%%%%%%%%%%%%%%%%
\newpage
\section*{課題1-2(発展)}
\myheader

\subsection*{説明}

ここに、課題(基礎)のソースコード、実行の様子の説明を記載。

\subsection*{考察}

ここに、課題(基礎)のソースコードのプログラミング手法や
コンピュータの動作などについて理解したこと、疑問に思ったこと、
残された課題と改善策などを記載。

\subsection*{ソースコード}

%\subsubsection*{helloworld.c}
%\lstinputlisting[basicstyle=\ttfamily\footnotesize, frame=single]{helloworld.c} % ソースファイルの読込

\subsection*{実行結果}

%\begin{quote}
%\begin{verbatim}

%\end{verbatim}
%\end{quote}

%%%%%%%%%%%%%%%%%%%%%%%%%%%%%%%%%%%%%%%%%%%%%%%%%%%%%%%%%
\newpage
\section*{課題1-3(応用)}
\myheader

\subsection*{説明}

ここに、課題(基礎)のソースコード、実行の様子の説明を記載。

\subsection*{考察}

ここに、課題(基礎)のソースコードのプログラミング手法や
コンピュータの動作などについて理解したこと、疑問に思ったこと、
残された課題と改善策などを記載。

\subsection*{ソースコード}

%\subsubsection*{helloworld.c}
%\lstinputlisting[basicstyle=\ttfamily\footnotesize, %frame=single]{helloworld.c}

\subsection*{実行結果}

%\begin{quote}
%\begin{verbatim}
%
%\end{verbatim}
%\end{quote}

\end{document}
